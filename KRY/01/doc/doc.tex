\documentclass[11pt,a4paper]{article}
\usepackage[left=2cm, text={17cm, 24cm}, top=2.5cm]{geometry}
\usepackage[czech]{babel}
\usepackage[utf8]{inputenc}
\usepackage{enumitem}
\usepackage{times}
\usepackage{float}
\usepackage[dvipsnames]{xcolor}
\usepackage{amsmath}
\usepackage{amssymb}
\usepackage{amsthm}
\usepackage{tikz}
\usepackage{fancyhdr}
\usepackage{listings}
\usepackage{algorithm}
\usepackage[noend]{algpseudocode}
\lstset{
	language=bash,
	basicstyle=\ttfamily
}

\usepackage{lastpage}

\usepackage{multicol}

\usepackage{tabularx}
\newcounter{tab}
\renewcommand*{\thetab}{(\arabic{tab})}

\pagestyle{fancy}
\setlength{\headheight}{28pt} 
\fancyhf{}
% jednostranná sazba
\fancyhead[L]{
	\begin{tabular}{lr}
		Jméno: & Karel \textsc{Ondřej} \\
		Login: & xondre09
	\end{tabular}
}
\fancyhead[C]{
	\begin{tabular}{c}
		\textbf{Kryptografie (KRY)}\\
		Projekt č. 1
	\end{tabular}
}
\fancyhead[R]{
	\begin{tabular}{lr}
		Datum: & \today \\
		Stránka: & \thepage/\pageref{LastPage}
	\end{tabular}
}

\begin{document}
\section{Kryptoanalýza}
Prvním krokem analýzy je zjistit co nejvíce informací o použitém šifrovacím algoritmu. Jednotlivé zašifrované zprávy $A$, $B$ lze vyjádřit pomocí vztahů \ref{eq:EA} a \ref{eq:EB}, $E$ je zašifrovaná zpráva a $C(K)$ je proudový klíč generovaný tajným klíčem $K$.
\begin{equation}\label{eq:EA}
	E(A) = A \oplus C(K)
\end{equation}
\begin{equation}\label{eq:EB}
	E(B) = B \oplus C(K)
\end{equation}
Dále můžeme využít vlastnosti operace exkluzivního součtu $\oplus$, která je komutativní a platí $X\oplus X = 0$:

\begin{equation}\label{eq:EAEB}
	E(A) \oplus E(B) = (A \oplus C(K)) \oplus (B \oplus C(K)) =  A \oplus B \oplus C(K) \oplus C(K) = A \oplus B\ .
\end{equation}
Jelikož známe obsah jedné ze zpráv, tak jednoduše dokážeme zjistit obsah druhé zprávy do délky známé zprá\-vy. 

Předpokládejme tedy, že byly všechny soubory zašifrovány stejným klíčem. Po aplikování vztahu \ref{eq:EAEB} a~znalosti zprávy \texttt{bis.txt} lze dešifrovat obsah první části zpráv \texttt{super\_cipher.py} a \texttt{hint.gif}.

Obrázek pravděpodobně může obsahovat dostatečně dlouhou posloupnost \uv{0}. Jelikož \uv{0} je neutrálním prvkem k operaci exkluzivního součtu $X\oplus 0 = 0 \oplus X = X$, tak po zašifrování souboru \texttt{hint.gif} se část zašifrovaného souboru \texttt{hint.gif.enc} stane klíčem $C(K)$, který lze použít k dešifrování další části souboru \texttt{super\_cipher.py.enc}. Po dešifrování části souboru \texttt{super\_cipher.py.enc} s použitím \texttt{hint.gif.enc} zbývá posledních 35 ne\-de\-šif\-rovaných bytů, které lze již snadno odhadnout.

\subsection{Nalezení použité substituce}

Po dešifrování přiloženého skriptu je potřeba ověřit, zda byl použit k zašifrování. To lze snadno ověřit tak, že vezmeme prvních 32 bytů $C(K)$ a zda dokážeme pomocí funkce \texttt{step} (viz výpis \ref{lst:step}) vygenerovat následujících 32 bytů $C(K)$. 

\begin{lstlisting}[language=python, caption=Generování proudového klíče $C(K)$ ze souboru \texttt{super\_cipher.py}., frame=single, captionpos=b, label=lst:step]
SUB = [0, 1, 1, 0, 0, 1, 0, 1]
N_B = 32
N = 8 * N_B
# Next keystream
def step(x):
    x = (x & 1) << N+1 | x << 1 | x >> N-1
    y = 0
    for i in range(N):
        y |= SUB[(x >> i) & 7] << i
    return y
\end{lstlisting}

Výsledkem testu je zjištění, že daný skript nebyl použitý k zašifrování souborů. Dalším krokem je tedy vyzkoušet, zda nebyla použita jiná substituce \texttt{SUB}. Jelikož je pouze $2^8 = 256$ možných kombinací, tak není zapotřebí vyloučit nesmyslné kombinace a lze vyzkoušet všechny. Pro jednotlivé substituce se postupně aplikuje funkce \texttt{step} na prvních 32 bytů $C(K)$, až se nalezne použitá substituce \texttt{SUB = [0, 1, 1, 0, 1, 0, 1, 0]}. 

Obě varianty řešení nejdříve získají klíč $C(K)$ ze souborů \texttt{bis.txt} a \texttt{bis.txt.enc} a následně naleznou použitou substituci. Na základě použité substituce se vytvoří inverzní funkce k generování proudového klíče $C(K)$, která se podle varianty řešení liší a je popsána v následujících kapitolách.

\section{Získání tajného klíče}
K funkci \texttt{step} lze sestrojit inverzní funkci, která získá předchozích 32 bytů $C(K)$. Proudový klíč $C(K)$ je inicializován $N/2$ aplikacemi funkce \texttt{step} na tajný klíč $K$. Po stejném počtu aplikací inverzní funkce na prvních 32 bytů $C(K)$ lze získat tajný klíč $K$. Následující kapitoly popisují inverzní funkce obou variant řešení.

\subsection{Ruční řešení}

Při ručním řešení se nejdříve sestrojí množiny $S_0$ a $S_1$ obsahující hodnoty, které byly při zašifrování substituovány za \uv{0} a \uv{1}. 
Na vstupu mějme $N$ bitů proudového klíče $C(K)$, se kterým budeme pracovat jako s~řetězcem $d\in\{0, 1\}^N$. Značení $d_n$ odpovídá n-tému znaku zprava (indexováno od 0), neboli n-tému nejnižšímu bitu. 

Mějme množinu všech doposud možných řešení $R$, inicializovanou jako $R_0 = \{0, 1\}^2$. 
Volba těchto hodnot je zdůvodněna níže. 
Postupně se iteruje přes jednotlivé symboly řetězce $d$, kde $i \in \mathbb{N} < N$ je pořadí iterace a $d_i \in \{0, 1\}$. 
Při každé iteraci se vypočte podmnožina kartézského součinu $K_i \subseteq S_{d_i} \times R_i$, kde $(s, r) \in K_i$ a dva nejpravější symboly řetězce $s$ jsou stejné jako dva nejlevější symboly řetězce $r$, tzn. $s_1.s_0 = r_{i+1}.r_{i}$.
Z vybraných n-tic se vytvoří nová množina $R_{i+1} = \{s_3.r\ |\ (s, r) \in K_i\}$, kde se nejlevější symbol $s_3$ zřetězí s~řetězcem $r$. 
Počáteční množina $R_0$ je zvolena právě tak, aby $R_1 = K_0$.

V poslední iteraci bude množina $R_N$ obsahovat všechny možné kombinace, při kterých po aplikaci substituce získáme vstupní řetězec $d$. Operace substituce pracuje s~cyklickým bufferem a porovnají se vždy 3~symboly, tudíž musí být dva nejlevější symboly rovny dvěma nejpravějším symbolům. Pokud byla použita platná substituce, tak tomuto kritériu bude odpovídat právě jedno řešení $y \in R_N$.  Hledaný řetězec potom odpovídá $x = y_{N+1}.y_N. \cdots .y_2.y_1$, jelikož před substitucí byla provedena operace rotace doleva o jeden symbol. Řetězec $x$ odpovídá binární reprezentaci předcházející posloupnosti proudového klíče $C(K)$.


\subsection{SAT solver}

Nalezená substituce lze chápat jako logická funkce o třech proměnných. Tato funkce lze jednoduše vyjádřit pomocí CNF. Pro každou sousedící trojici bitů (nejvyšší bit sousedí s nejnižším) se sestrojí rovnice v CNF, jejíž výsledek odpovídá zvolené nalezené substituci. V těchto rovnicích se již počítá s počáteční rotací $N$ bitů vstupního $C(K)$.

\emph{SAT solver} následně hledá kombinaci vstupních $N$ proměnných tak, aby výsledky všech rovnic odpovídaly hodnotě bitu v $C(K)$, na který byla trojice substituována. Vztahy \ref{eq:p1}, \ref{eq:p2}, \ref{eq:p4}, \ref{eq:p6} a \ref{eq:SAT} znázorňují soustavu rovnic a~predikátů pro substituci \texttt{SUB = [0, 1, 1, 0, 1, 0, 1, 0]}.

\begin{align}\label{eq:p1}
	p_1(i) &\equiv x_{i+1\bmod N} \wedge \neg x_{i+2\bmod N}  \wedge \neg x_{i+3\bmod N} \\\label{eq:p2}
	p_2(i) &\equiv \neg x_{i+1\bmod N} \wedge x_{i+2\bmod N}  \wedge \neg x_{i+3\bmod N} \\\label{eq:p4}
	p_4(i) &\equiv \neg x_{i+1\bmod N} \wedge \neg x_{i+2\bmod N}  \wedge x_{i+3\bmod N} \\\label{eq:p6}
	p_6(i) &\equiv \neg x_{i+1\bmod N} \wedge x_{i+2\bmod N}  \wedge x_{i+3\bmod N}
\end{align}
\begin{equation}\label{eq:SAT}
\bigwedge_{i=0}^{N-1} p_1(i) \vee p_2(i) \vee p_4(i) \vee p_6(i) = d_i
\end{equation}

V implementaci byla zvolena množina proměnných typu \texttt{Bool}, kde každá proměnná reprezentuje jeden bit, místo typu \texttt{VecBit} reprezentující celé číslo. Řešení s použitím \texttt{VecBit} bylo časově náročnější než použití množiny proměnných.

\section{Závěr}

V rámci projektu byla provedena analýza zadaných souborů, která vedla k odhalení použitého tajného klíče \uv{KRY\{xondre09-28ff1748ad3563d\}}. K tomu bylo využito: únik nezašifrovaného textu, opakované použití stejného tajného klíče a znalost použitého algoritmu. 

Řešení postupně ze zadaných souborů naleznou generovaný proudový klíč a použitou substituci při jeho generování. Při ručním řešení je vytvořen algoritmus, který provádí reverzní substituci a v druhém řešení se sestaví rovnice, které následně vyřeší \emph{SAT solver}. Čas pro nalezení tajného klíče trvá na procesoru Intel Core i7-4810MQ CPU 2.80GHz pro \emph{SAT solver} 5 vteřin a pro ruční řešení téměř okamžitě.


\end{document}  

